\documentclass{beamer}
\usetheme{AnnArbor}
\usecolortheme{beaver}
\usepackage{tikz}
\usepackage{color}
\usepackage{listings}

\lstset{language=Java,
  basicstyle=\footnotesize\ttfamily,
  keywordstyle=\footnotesize\color{blue}\ttfamily,
  commentstyle=\footnotesize\color{gray}\ttfamily,
}

\definecolor{darkred}{rgb}{0.8,0,0}

\setbeamercolor{title}{fg=white,bg=darkred!80!black}
\setbeamercolor{frametitle}{fg=darkred!80!black,bg=white}
%\setbeamercolor{section in head/foot}{fg=green,bg=yellow}
%\setbeamercolor{subsection in head/foot}{bg=white}
\begin{document}
\title{Getting started with RedDeer}   
\author{Andrej Podhradsky}
\date{\today} 
%\logo{\includegraphics[height=1cm]{reddeer_logo.png}\vspace{220pt}}

\addtobeamertemplate{title page}{\center{\includegraphics[height=2cm]{reddeer_logo.png}}}{}

\addtobeamertemplate{frametitle}{}{
\begin{tikzpicture}[remember picture,overlay]
\node[anchor=north east,yshift=-8pt] at (current page.north east) {\includegraphics[height=1cm]{reddeer_logo.png}};
\end{tikzpicture}}

\frame{\titlepage} 

\frame{\frametitle{Table of contents}\tableofcontents} 


\section{Introduction}
\subsection{RedDeer Introduction}
\begin{frame}[fragile]
\frametitle{RedDeer Introduction}
\begin{itemize}
\item Framework for automating eclipse tests\\\url{https://github.com/jboss-reddeer/reddeer/wiki}
\item Tests are built and executed by tycho plugin (as all eclipse plugins)
\item Example
\begin{lstlisting}
new PushButton("OK").click();
new LabeledText("Name").setText("test-name");
new TextEditor("Demo.java").save();
\end{lstlisting}
\item How it works?
\begin{itemize}
\item Each component is found in the constructor
\item The appropriate method (or event) is called in the main thread
\item Waiting until the operation has finished
\end{itemize}
\end{itemize}
\end{frame}

\subsection{RedDeer Hi-Level API}
\begin{frame}[fragile]
\frametitle{RedDeer Hi-Level API}
\begin{itemize}
\item This is key API for building functional tests by using RedDeer
\item Wizards
\begin{lstlisting}
NewJavaClassWizardDialog wizard =
        new NewJavaClassWizardDialog();
wizard.open();
wizard.getFirstPage().setName("HelloWorld");
wizard.getFirstPage().setPackage("com.example");
wizard.finish();
\end{lstlisting}
\item Views
\begin{lstlisting}
ProblemsView problemsView = new ProblemsView();
problemsView.open();
problemsView.getAllErrors();
problemsView.getAllWarnings();
\end{lstlisting}
\end{itemize}
\end{frame}


\section{Test Execution}

\subsection{Prerequisite}
\begin{frame}[fragile]
\frametitle{Prerequisite}
\begin{itemize}
\item Eclipse Luna for JEE Developers or JBDS 7.1.1.GA
\item RedDeer v0.6.0\\\url{http://download.jboss.org/jbosstools/updates/stable/luna/core/reddeer/0.6.0}
\item Display server, e.g. vnc\\We need isolated display since other events can affect our tests
\begin{lstlisting}[language=sh]
sudo yum install tigervnc tigervnc-server
vncpasswd    // set password
vncserver :2 -geometry 1500x900
vncviewer :2
\end{lstlisting}
\item Each test class must be annotated with
\begin{lstlisting}[language=Java]
@RunWith(RedDeerSuite.class)
\end{lstlisting}
\end{itemize}
\end{frame}

\subsection{Using Maven}
\begin{frame}[fragile]
\frametitle{Using Maven}
\begin{itemize}
\item Build reddeer archetype
\begin{lstlisting}[language=sh]
git clone https://github.com/jboss-reddeer/reddeer.git
mvn clean install -pl archetype/ -DskipTests
\end{lstlisting}
\item Create a reddeer project
\begin{lstlisting}[language=sh]
mvn archetype:generate 
        -DarchetypeGroupId=org.jboss.reddeer
        -DarchetypeArtifactId=jboss-reddeer-archetype
\end{lstlisting}
\item Execute tests
\begin{lstlisting}[language=sh]
DISPLAY=:2 mvn clean verify [-Dtest...] [-Dtycho.localArtifacts=ignore]
\end{lstlisting}
\item Executing tests against an existing eclipse instance
\begin{lstlisting}[language=sh]
DISPLAY=:2 mvn clean verify -Dtest.installPath=...
\end{lstlisting}
\end{itemize}
\end{frame}

\subsection{Using IDE}
\begin{frame}[fragile]
\frametitle{Using IDE}
\begin{enumerate}
\item Import existing project as maven project
\item Select Run \textgreater Run Configurations...
\item Create new RedDeer Test configuration
\item On tab Environment add new variable DISPLAY with value :2
\end{enumerate}
\end{frame}


\section{Debugging}
\begin{frame}[fragile]
\frametitle{Debugging}
\begin{itemize}
\item You can debug RedDeer tests as any other JUnit tests
\item Remote debugging (on localhost, jenkins, ...)
\begin{lstlisting}[language=sh]
DISPLAY=:2 mvn clean verify -DdebugPort=8001
\end{lstlisting}
Then, in IDE do as follows
\begin{enumerate}
\item Select Run \textgreater Debug Configurations...
\item Create new Remote Java Application
\item Select Standard (Socket Attach) as connection type
\item Set package, host and port (8001)
\end{enumerate}
\end{itemize}
\end{frame}


\section{RedDeer in Details}

\subsection{Matchers}
\begin{frame}[fragile]
\frametitle{Matchers}
\end{frame}

\subsection{Conditions}
\begin{frame}[fragile]
\frametitle{Conditions}
\end{frame}

\subsection{Waits and timeouts}
\begin{frame}[fragile]
\frametitle{Waits and timeouts}
\end{frame}

\end{document}
