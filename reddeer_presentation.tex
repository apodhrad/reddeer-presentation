\documentclass{beamer}
\usetheme{AnnArbor}
\usecolortheme{beaver}
\usepackage{tikz}
\usepackage{color}
\usepackage{listings}

\lstset{language=Java,
  basicstyle=\footnotesize\ttfamily,
  keywordstyle=\footnotesize\color{blue}\ttfamily,
}

\definecolor{darkred}{rgb}{0.8,0,0}

\setbeamercolor{title}{fg=white,bg=darkred!80!black}
\setbeamercolor{frametitle}{fg=darkred!80!black,bg=white}
%\setbeamercolor{section in head/foot}{fg=green,bg=yellow}
%\setbeamercolor{subsection in head/foot}{bg=white}
\begin{document}
\title{Getting started with RedDeer}   
\author{Andrej Podhradsky}
\date{\today} 
%\logo{\includegraphics[height=1cm]{reddeer_logo.png}\vspace{220pt}}

\addtobeamertemplate{title page}{\center{\includegraphics[height=2cm]{reddeer_logo.png}}}{}

\addtobeamertemplate{frametitle}{}{
\begin{tikzpicture}[remember picture,overlay]
\node[anchor=north east,yshift=-8pt] at (current page.north east) {\includegraphics[height=1cm]{reddeer_logo.png}};
\end{tikzpicture}}

\frame{\titlepage} 

\frame{\frametitle{Table of contents}\tableofcontents} 

\section{Introduction}
\subsection{RedDeer Introduction}
\begin{frame}[fragile]
\frametitle{RedDeer Introduction}
\begin{itemize}
\item Framework for automating eclipse tests\\\url{https://github.com/jboss-reddeer/reddeer/wiki}
\item Update site\\\url{http://download.jboss.org/jbosstools/updates/stable/luna/core/reddeer/0.6.0}
\item Example
\begin{lstlisting}
new PushButton("OK").click();
new LabeledText("Name").setText("test-name");
new TextEditor("Demo.java").save();
\end{lstlisting}
\item How it works?
\begin{itemize}
\item Each component is found in the constructor
\item The appropriate method (or event) is called in the main thread
\item Waiting until the operation has finished
\end{itemize}
\end{itemize}
\end{frame}

\subsection{RedDeer Hi-Level API}
\begin{frame}[fragile]
\frametitle{RedDeer Hi-Level API}
\begin{itemize}
\item This is key API for building functional tests by using RedDeer
\item Wizards
\begin{lstlisting}
NewJavaClassWizardDialog wizard =
        new NewJavaClassWizardDialog();
wizard.open();
wizard.getFirstPage().setName("HelloWorld");
wizard.getFirstPage().setPackage("com.example");
wizard.finish();
\end{lstlisting}
\item Views
\begin{lstlisting}
ProblemsView problemsView = new ProblemsView();
problemsView.open();
problemsView.getAllErrors();
problemsView.getAllWarnings();
\end{lstlisting}
\end{itemize}
\end{frame}

\subsection{RedDeer Test Execution}
\begin{frame}[fragile]
\frametitle{RedDeer Test Execution}
\begin{itemize}
\item Generate reddeer maven archetype
\begin{lstlisting}[language=sh]
git clone https://github.com/jboss-reddeer/reddeer.git
mvn clean install -pl archetype/ -DskipTests
\end{lstlisting}
\item Create a reddeer project
\begin{lstlisting}[language=sh]
mvn archetype:generate 
        -DarchetypeGroupId=org.jboss.reddeer
        -DarchetypeArtifactId=jboss-reddeer-archetype
mvn clean verify
\end{lstlisting}
\item Notice that each test class is annotated with
\begin{lstlisting}[language=Java]
@RunWith(RedDeerSuite.class)
\end{lstlisting}
\item From IDE just select Run As \textgreater RedDeer Test
\item It's recommended to run tests in another display (vnc, xephyr, ...)
\end{itemize}
\end{frame}

\section{Requirements}
\frame{\frametitle{What you need}
\begin{itemize}
\item Eclipse Luna / JBDS 7+ \pause
\item VNC server / viewer \pause
\end{itemize}
}


\section{SWT widgets and how to find them} 
\subsection{SWT widgets}
\frame{\frametitle{SWT widgets}
}
\subsection{How to find widgets}
\frame{\frametitle{How to find widgets}
}

\section{Executing RedDeer tests}
\subsection{From IDE}
\frame{\frametitle{From IDE}
}
\subsection{Using maven}
\frame{\frametitle{Using maven}
}


\section{Example}
\subsection{Example}
\begin{frame}[fragile]
\frametitle{Example}
\begin{lstlisting}
public class HelloWorld {
  public static void main(String[] args) {
    System.out.println("Hello World");
  }
}
\end{lstlisting}
\end{frame}

\section{Debugging}
\subsection{Debugging}
\frame{\frametitle{Debugging}
}

\end{document}
